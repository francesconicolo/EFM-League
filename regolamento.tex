%PREAMBOLO
\documentclass[a4paper, 12pt]{report}
\usepackage[italian]{babel}
\usepackage{graphicx}
\usepackage{amsmath,amssymb}
\usepackage{amsbsy}
\usepackage{xcolor}
\usepackage{enumitem}
\usepackage{multicol}
\usepackage{geometry}
\usepackage{background}
\renewcommand{\footnoterule}{
  \kern -3pt
  \hrule width \textwidth height 1pt
  \kern 2pt
  
}%ALLUNGA LINEA PIE DI PAGINA
\backgroundsetup{
scale=0.8,
color=black,
opacity=0.08,
angle=0,
contents={%
  \includegraphics[width=\paperwidth]{Logocentrato.png}
  }%
}

%INIZIO
\begin{document}
\title{\textbf{REGOLAMENTO}}
\author{"Easy For Me" League}
\date{\today}
\maketitle
\tableofcontents
\chapter{Introduzione}
\section{Panoramica}
\paragraph{"Easy For Me" League} è una competizione agonistica, fondata nel 2023 da un gruppo di ragazzi appassionati di videogiochi competitivi;
\chapter{Regolamento}
\paragraph{}Il regolamento della EFM League prevede un regolamento breve, semplice, ma categorico. Le partite verranno considerate valevoli per la lega dall'11/09/23.
\section*{Regole}
\begin{enumerate}
  \item I membri possono partecipare alla custom SE E SOLO SE sono presenti anche su discord.
  \item I membri del team vincente guadagneranno 1 PUNTO.
  \item I membri del team perdendo guadagneranno 0 PUNTI.
  \item Quando la partita viene considerata annullata non verranno assegnati punteggi.
  \item L'assegnazione di 0 o meno punti vengono considerati come "SCONFITTA" per il giocatore.
  \item L'assegnazione di più di 0 punti verrà considerata come "VITTORIA" per il giocatore.
  \item Chi riceverà 2 AMMONIZIONI subirà una penalizzazione di -1 PUNTO.
  \item Presentarsi in ritardo (sopra i 40 minuti verrà applicata la regola n.8), comporterà una possibile ammonizione che sarà decisa di comune accordo tra i membri già online, si consiglia di avvisare l'eventuale ritardo.
  \item Votare SI nel sondaggio e non presentarsi alla custom comporterà una possibile esclusione temporanea dalla lega e una penalità di punti.
  \item La votazione del sondaggio può essere cambiata al massimo un'ora prima dell'ora stabilita.
  \item Si possono chiamare un massimo di 4 pause a partita per ogni team (salvo comune accordo).
  \item Il team che chiamerà più di 4 pause perderà la partita a tavolino, con la regolare assegnazione dei punteggi.
  \item Se la Champ Select non dovesse corrispondere alla draft precedentemente svolta, la partita sarà considerata nulla e sarà ripetuta, il giocatore che ha sbagliato sarà ammonito.
  \item Se sono presenti più di 2 forestieri (persone non incluse nella lista) nella custom, la partita non sarà considerata valevole per la EFM League.
  \item Durante la draft chi non avviserà entro massimo 10 secondi la correzione del pick o ban, nel caso in cui si fosse sbagliato, non potrà cambiare la sua decisione.
  \item In caso di abbandono di un membro del team, che ha un vantaggio di almeno 5k gold, il giocatore che abbandonerà subirà una penalità di -1 PUNTO e i restanti membri del team potranno decidere se continuare la partita o pattare, dando un punteggio di +0 a tutti i presenti.
  \item In caso di abbandono di un membro del team, che ha uno svantaggio di almeno 5k gold, il giocatore che abbandonerà subirà una penalità di -1 punto, i restanti membri del team potranno decidere di arrendersi in anticipo prendendo un punteggio di +0. Il team avversario prenderà regolarmente +1 punto.
  \item In caso di abbandono di un membro di un qualsiasi team quando lo svantaggio tra le due squadre dovesse essere inferiore ai 5k gold, il giocatore subirà una penalità di -1 PUNTO e la partita sarà annullata.
  \item Se in seguito a votazione, per usare la ruota, si raggiunge la maggioranza semplice ((partecipanti/2) +1) la partita sarà considerata valevole per la EFM League.
  \item Raz dovrà farsi perdonare in tutti i modi possibili per essere ammesso alla EFM League
\end{enumerate}
\end{document}
